%!TeX program = pdflatex

% Beispieldatei, auskommentieren in finaler Version!
\begin{filecontents*}{daten.csv}
	Sonnenschein;Summer;w;Wetterstraße;5;1111;Strahlenau
	Quietsch;York;m;Bachstr.;4;67345;Ententeich
	Firma Anwander Gmbh;;;Wattstr.;11;1111;Turgau
\end{filecontents*}


\documentclass[%
fontsize=12pt, % Schriftgröße
version=last, % Neueste Version von KOMA-Skript verwenden
parskip=full % Kein hässliches Einrücken
]{scrlttr2}

% ===== Deutsche Sprache =====
\usepackage[utf8]{inputenc}
\usepackage[ngerman]{babel}
% ============================

\LoadLetterOption{DIN} % Einstellungen für DIN 676 laden
\LoadLetterOption{invitationoptions} % Absenderdaten und -einstellungen aus absender.lco laden

% Einstellungen für CSV
\usepackage{datatool}
\DTLsetseparator{;}
\DTLloaddb
[noheader,
keys={name,vorname,gender,strasse,nr,plz,ort}
]
{adressen}
{daten.csv}

\begin{document}

\DTLforeach{adressen}
{\Name=name,%
	\Vorname=vorname,%
	\Gender=gender,%
	\Str=strasse,%
	\Nr=nr,%
	\PLZ=plz,%
	\Ort=ort%
}{
	\begin{letter}%
{\ifstr{\Gender}{}{\Name \\ \Str~\Nr \\ \PLZ~\Ort}{\Vorname~\Name \\ \Str~\Nr \\ \PLZ~\Ort}}

\setkomavar{date}{\today}       % Datum
% =====================================

% Eingestellt über die invitationoptions.lco
%\setkomavar{title}{Titel}
%\setkomavar{subject}{Betreff}

%\opening{Sehr geehrte Frau Musterfrau,}
\opening{\ifstr{\Gender}{w}{Sehr geehrte Frau \Name,}{\ifstr{\Gender}{m}{Sehr geehrter Herr \Name,}{\ifstr{\Gender}{}{Sehr geehrte Damen und Herren,}{}}}}


\input{invitation.txt}

%\closing{Mit freundlichen Grüßen}

\end{letter}
}
\end{document}
